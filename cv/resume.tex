\documentclass[10pt,a4paper,withhyper]{altacv}
%% AltaCV uses the fontawesome5 and simpleicons packages.
%% See http://texdoc.net/pkg/fontawesome5 and http://texdoc.net/pkg/simpleicons for full list of symbols.

% Change the page layout if you need to
\geometry{left=1.25cm,right=1.25cm,top=1.5cm,bottom=1.5cm,columnsep=1.2cm}

% The paracol package lets you typeset columns of text in parallel
\usepackage{paracol}
% the hyperref package is used for creating hyperlinks
\usepackage{hyperref}

% Change the font if you want to, depending on whether
% you're using pdflatex or xelatex/lualatex
% WHEN COMPILING WITH XELATEX PLEASE USE
% xelatex -shell-escape -output-driver="xdvipdfmx -z 0" mmayer.tex
\iftutex
  % If using xelatex or lualatex:
  \setmainfont{SF Pro Text}
\else
  % If using pdflatex:
  \usepackage[default]{lato}
\fi

% Change the colours if you want to
\definecolor{VividPurple}{HTML}{1E3F81}
\definecolor{SlateGrey}{HTML}{2E2E2E}
\definecolor{LightGrey}{HTML}{666666}
\colorlet{accent}{blue!70!black}
\colorlet{emphasis}{black}
\colorlet{heading}{black!70!blue}
\colorlet{headingrule}{black}
\colorlet{subheading}{emphasis}
\colorlet{body}{black!80!white}

\colorlet{tagline}{blue!35!black}
\colorlet{name}{blue!35!black}
\colorlet{heading}{VividPurple}
\colorlet{headingrule}{VividPurple}
\colorlet{accent}{VividPurple}
\colorlet{emphasis}{SlateGrey}
\colorlet{body}{LightGrey}


% Change some fonts, if necessary
% \renewcommand{\namefont}{\Huge\rmfamily\bfseries}
% \renewcommand{\personalinfofont}{\footnotesize}
% \renewcommand{\cvsectionfont}{\LARGE\rmfamily\bfseries}
% \renewcommand{\cvsubsectionfont}{\large\bfseries}

% Change the bullets for itemize and rating marker
% for \cvskill if you want to
\renewcommand{\cvItemMarker}{{\small\textbullet}}
\renewcommand{\cvRatingMarker}{\faCircle}
% ...and the markers for the date/location for \cvevent
% \renewcommand{\cvDateMarker}{\faCalendar*[regular]}
% \renewcommand{\cvLocationMarker}{\faMapMarker*}

\begin{document}
\name{Laurent Brusa}
\tagline{Software Developer and Proud Geek}
%% You can add multiple photos on the left or right
\photoR{3cm}{laurent-gemini.png}
\personalinfo{%
  % Not all of these are required!
  \vspace{0.2em} 
  \email{laurentbrusa@me.com}
  \phone{+49 152 2962 3248}
  \location{Berlin, DE}
  \linkedin{laurentbrusa}\\[0.5em]
  \homepage{laurentbrusa.hashnode.dev}
  \github{multitudes}
  \NewInfoField{stackoverflow}{\faStackOverflow}[https://stackoverflow.com/users/9497800/multitudes?tab=profile]
  \stackoverflow{multitudes}

  %% You can add your own arbitrary detail with
  %% \printinfo{symbol}{detail}[optional hyperlink prefix]
  %% \printinfo{\faPaw}{Hey ho!}
  %% Or you can declare your own field with
  %% \NewInfoFiled{fieldname}{symbol}[optional hyperlink prefix] and use it:
  % \NewInfoField{gitlab}{\faGitlab}[https://gitlab.com/]
  % \gitlab{your_id}
	%%
  %% For services and platforms like Mastodon where there isn't a
  %% straightforward relation between the user ID/nickname and the hyperlink,
  %% you can use \printinfo directly e.g.
  % \printinfo{\faMastodon}{@username@instace}[https://instance.url/@username]
  %% But if you absolutely want to create new dedicated info fields for
  %% such platforms, then use \NewInfoField* with a star:
  % \NewInfoField*{mastodon}{\faMastodon}
  %% then you can use \mastodon, with TWO arguments where the 2nd argument is
  %% the full hyperlink.
  % \mastodon{@username@instance}{https://instance.url/@username}

  \vspace{0.5em} 
}

\makecvheader

%% Depending on your tastes, you may want to make fonts of itemize environments slightly smaller
\AtBeginEnvironment{itemize}{\small}

%% Set the left/right column width ratio to 6:4.
\columnratio{0.6}

% Start a 2-column paracol. Both the left and right columns will automatically
% break across pages if things get too long.
\begin{paracol}{2}
% Add extra vertical space here

\cvsection{Work Experience}

% \cvevent{Technical Assistant (part-time student occupation)}{FabLab Neukölln}{2024}{Berlin}
% \begin{itemize}
% \item Supported members with CNC machines, 3D printers, laser cutters
% \item Assembled custom hardware controllers (SOROTEC)
% \item Operated autonomously in a maker lab
% \end{itemize}
% \divider

\cvevent{iOS Developer}{\href{https://www.rooom.com}{Rooom}}{2021--2022}{Berlin}
\begin{itemize}
\item Built immersive AR apps for exhibitions like 
\item Managed full development lifecycle for iOS projects
\item Collaborated across teams and responded to user needs
\end{itemize}
\divider

\cvevent{iOS Developer (Internship)}{\href{https://tim-deussen.de}{Studio Tim Deussen}}{2020}{Berlin}
\begin{itemize}
\item Developed AR applications using ARKit
\item Collaborated with designers to create user-friendly interfaces
\item Participated in code reviews and contributed to team knowledge sharing
\end{itemize}
\divider

\cvevent{Senior Travel Consultant}{Apple, AirFrance, CWT, ANA}{2001--2018}{London and Berlin}
\begin{itemize}
\item Handled multilingual corporate support in high-pressure environments
\item Developed strong user empathy, clear communication, and prioritization skills
\end{itemize}

\divider

\cvsection{Projects and Apps}

\cvevent{\faGithub~\href{https://multitudes.github.io/42-transcendence/}{Transcendence} (Web Multiplayer game)}{}{2025}{42 Berlin}
\begin{itemize}
  \item Deployed full Docker Compose stack with monitoring via ELK and Grafana, backend in Django and frontend as a single-page application using pure vanilla JavaScript for frontend
  \item Designed microservices deployment and implemented GitHub Action pipelines for testing.
\end{itemize}

\divider

\cvevent{\faGithub~\href{https://multitudes.github.io/httpwebserver/}{HTTP Web Server}}{}{2024}{42 Berlin}
\begin{itemize}
  \item “What I cannot create, I do not understand.” ~ Richard Feynman
  \item Handles multiple connections via multiplexing (poll/select), serves static files, processes various HTTP methods, and supports CGI scripts
\end{itemize}

\divider


\cvevent{\faGithub~\href{https://multitudes.github.io/42-minishell/}{minishell} (custom shell)}{}{2023--2025}{42 Berlin}
\begin{itemize}
  \item Coded a full functional shell inspired by Bash without using external libraries.
  \item The command line interpreter uses its own tokenizer and tree logic to execute pipes and buildin commands.
  \item Supports parenthesis and "\&\&" and "||" operators. Redirections and heredocs.
\end{itemize}

\divider

\cvevent{\href{https://apps.apple.com/de/app/grünebandbreite/id1557697472?l=en-GB}{Das Grüne Band / The green belt app}}{App Store}{2022-2025}{iOS/Swift}
\begin{itemize}
  \item An iOS app for the exhibition at the Egapark in Erfurt (Germany)
  \item Showcases the history and significance of the former inner-German border
\end{itemize}

\divider

\cvevent{\href{https://wwdcnotes.com/documentation/wwdcnotes/multitudes}{WWDCNotes}}{}{2020-2025}{iOS/Swift}
\begin{itemize}
  \item An open-source app to collect notes for all Apple's WWDC videos
  \item Summarized and reviewed developer talks for the Swift community
\end{itemize}
\divider

\cvevent{\faGithub~\href{https://github.com/multitudes/NiceWeather/blob/main/README.md}{NiceWeather}}{}{2021}{iOS/Swift}
\begin{itemize}
  \item A weather app providing real-time weather updates and forecasts
  \item Utilized SwiftUI for a modern and responsive user interface
\end{itemize}
\divider


\cvevent{Airlock App}{}{2020}{iOS/Swift}
\begin{itemize}
  \item A meditation and journal app for iOS, designed to help users relax and reflect
  \item Coded with SwiftUI and CoreData
\end{itemize}
\divider

% \cvevent{Product Engineer}{Google}{23 June 1999 -- 2001}{Palo Alto, CA}

% \begin{itemize}
% \item Joined the company as employe \#20 and female employee \#1
% \item Developed targeted advertisement in order to use user's search queries and show them related ads
% \end{itemize}

% % use ONLY \newpage if you want to force a page break for
% % ONLY the currentc column
% \newpage


% |||||||||||||||||||||||||||||||||||||
\switchcolumn

\cvsection{Life Philosophy}
\begin{quote}
``The flow of the river is ceaseless and its water is never the same.''
\end{quote}

\vspace{1em} % Add extra vertical space here

%%%%%%%%%%%%%%%%%%%%%%%%%%%%%%%%
\cvsection{Most Proud of}

\cvachievement{\faHandsHelping}{Community Volunteer}{Supported tech events and developer communities in Berlin and online}
\cvachievement{\faHeartbeat}{Persistence}{to start a new career as ios deve, and then go on to learn C at 42 Berlin}
\cvachievement{\faBicycle}{Horizons}{Happy to have expanded my horizons cycling in many remote and less remote parts of the world}

\vspace{1em} % Add extra vertical space here


% ====================
\cvsection{Skills}

% Soft skills

\cvtag{Communication}
\cvtag{Teamwork}
\cvtag{Problem-solving}
\cvtag{Adaptability}

\divider

% Technical skills
\cvtag{Python}
\cvtag{Bash}
\cvtag{C}
\cvtag{Swift}
\cvtag{SQL}
\cvtag{LaTeX}
\cvtag{Docker}
\cvtag{Docker Compose}
\cvtag{Kubernetes (basics)}
\cvtag{Linux sysadmin}
\cvtag{RedHat Linux}
\cvtag{Grafana}
\cvtag{ELK Stack}
\cvtag{Git}
\cvtag{JSON APIs}
\cvtag{Pandas}
\cvtag{NumPy}
\cvtag{Postgres}
\cvtag{CLI}
\cvtag{LangFlow}
\cvtag{n8n}

\vspace{1em} % Add extra vertical space here
% =================================
\cvsection{Languages}

\cvskill{English (C1)}{5}
\vspace{0.3em} 
\cvskill{German (C1)}{5} %% supports X.5 values.
\vspace{0.3em} 
\cvskill{French C1}{5} %% supports X.5 values.
\vspace{0.3em} 
\cvskill{Italian (C1)}{5} %% supports X.5 values.
\vspace{0.3em} 
\cvskill{Spanish (B1)}{3}
\vspace{0.3em} 
\cvskill{Dutch (B1)}{3}
\vspace{0.3em} 
\cvskill{Japanese (B1)}{2,5}

\newpage

% ===================== second page ==========
\cvsection{Certificates}
\cvevent{Red Hat Academy Course}{}{2025}{}
\begin{itemize}
\item \href{https://www.credly.com/badges/4f5f6b86-c9ec-45d9-86ae-5cbb1d991d3e/linked_in_profile}{Red Hat Enterprise Linux Automation with Ansible (RH294 - RHA) - Ver. 9.0}
\end{itemize}
\divider

\cvevent{Red Hat Academy Course}{}{2025}{}
\begin{itemize}
\item \href{https://www.credly.com/badges/28664aa9-2d8e-4b69-94be-2357f34806e5/linked_in_profile}{Red Hat System Administration II (RH134 - RHA) - Ver. 9.3}
\end{itemize}
\divider

\cvevent{Bosch Hackathon}{}{2025}{}
\begin{itemize}
\item Certificate of participation for the 12h Agentic AI Hackathon
\end{itemize}

\divider

\cvevent{Coursera}{}{}{}
\begin{itemize}
  \item \href{https://www.coursera.org/account/accomplishments/certificate/4EHEGSJ3746D}{Google Cloud Big Data and Machine Learning
Fundamentals}
\end{itemize}

\divider

\cvevent{Test Automation University}{}{2020}{}
\begin{itemize}
\item \href{https://testautomationu.applitools.com/certificate/?id=b5472478}{Introduction to iOS Test Automation with XCUITest}
\end{itemize}

\divider


\cvsection{Volunteering}

\cvevent{Volunteer}{WeAreDevelopers Conference}{Berlin, 2025}{}
\begin{itemize}
\item Logistics and attendee support
\end{itemize}
\divider

\cvevent{Volunteer}{WeAreDevelopers Conference}{Berlin, 2024}{}
\begin{itemize}
\item Logistics and attendee support
\end{itemize}
\divider

\cvevent{Contributor}{WWDCNotes}{2023}{Remote}
\begin{itemize}
\item Summarized and reviewed developer talks for the Swift community
\end{itemize}
\divider

\cvevent{Remote Conference Support}{UIKonf}{2020}{Remote}
\begin{itemize}
\item Moderated sessions and supported attendees during pandemic-era remote conference
\end{itemize}
\divider

\switchcolumn


\cvsection{Education}


\cvevent{Specialisation Curriculum}{42 Berlin – Advanced Coding School}{2025 -- present}{Berlin}
\begin{itemize}
  \item Following the specialisation track for the RNCP 6 and 7 certifications
\end{itemize}
\divider

\cvevent{Core Curriculum}{42 Berlin – Advanced Coding School}{2023 -- 2025}{Berlin}
\begin{itemize}
  \item Focus on C, C++, Docker and system administration
\end{itemize}
\divider

\cvevent{IHK Software Developer Certificate}{IHK}{2018 -- 2020}{Berlin}


% ============ maybe add later
% \cvsection{Referees}

% % \cvref{name}{email}{mailing address}
% \cvref{Prof.\ Alpha Beta}{Institute}{a.beta@university.edu}
% {Address Line 1\\Address Line 2}

% \divider

% \cvref{Prof.\ Gamma Delta}{Institute}{g.delta@university.edu}
% {Address Line 1\\Address Line 2}

\end{paracol}

\end{document}
